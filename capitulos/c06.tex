\chapter{ Conclusões }

Para desenvolver o projeto foi escolhido o software que até a presente
data é o mais recomendado para tarefas de aprendizado de máquina. Um
reconhecedor de texto em imagens de CAPTCHA foi desenvolvido ao longo
do trabalho. Este reconhecedor conta com um alto nível de robustez
diante dos testes realizados.

Os testes realizados em todos os casos mostraram ser possível atingir um
resultado razoável na tarefa de reconhecimento de textos em imagens,
isso com poucos ajustes à configuração de treinamento de redes
neurais. Atualmente a quantidade de exemplos e tutoriais disponíveis
para tarefas de aprendizado de máquina é imenso. Fica claro que é
possível implementar classificadores mesmo com poucos recursos.

\section{Vulnerabilidade de fontes públicas}

Diante do objetivo alcançado pelo trabalho, fica aparente que fontes
públicas de dados podem estar vulneráveis. Consultas automatizadas
realizadas por \textit{Web crawlers} podem não afetar diretamente a
segurança das informações, isso porque as informações já estão sendo
disponibilizadas publicamente. Entretanto a disponibilidade de tais
fontes pode ser afetada quando o ambiente de um \textit{website} não
está preparado para um volume muito grande de consultas.

\subsection{Eficácia de CAPTCHAs}

Enquanto é possível discutir a eficácia dessas imagens de CAPTCHA e
como gerar imagens mais difíceis, também cabe uma discussão sobre a
necessidade de imagens na tentativa de bloquear consultas
automatizadas. Imagens usadas como CAPTCHA geralmente são frustrantes
para usuários humanos de sistemas de consulta. Ao tentar dificultar o
reconhecimento de imagens por máquinas, também se dificulta o acesso
de um usuário comum às informações. Portanto é possível abrir espaço
para estudos que buscam outras formas de bloqueio de \textit{Web
  crawlers}. 

Outra alternativa, inclusive mais interessante, seria
disponibilizar outros tipos de consulta específicos para sistemas de
terceiros que desejam utilizar dados públicos. Assim um
\textit{website} de fonte pública de dados poderia continuar a servir
usuários humanos com robustez e ao mesmo tempo servir usuários
sistêmicos com um formato mais adequado.

\section{Trabalhos futuros}

Como possíveis trabalhos futuros, cita-se: 

\begin{itemize}

        \item Fazer um melhor uso das informações geradas pelo
          processo de treinamento para gerar heurísticas mais
          inteligentes. Um exemplo seria utilizar outros tipos de
          otimizadores para a função da perda.

        \item Estender o sistema para realizar o reconhecimento de
          outros tipos de CAPTCHAs.

        \item Estender o sistema para realizar o reconhecimento de
          tipos de CAPTCHAs que possuem um tamanho de texto variável.

        \item Realizar um estudo sobre \textit{Web crawlers} em fontes
          públicas que utilizam CAPTCHA utilizando o sistema proposto
          neste trabalho.

        \item Implementar um sistema de reconhecimento de CAPTCHAs
          mais avançados que solicitam a classificação de uma cena
          completa ou identificação de objetos em imagens.

        \item Estudar um artifício mais efetivo para o bloqueio de
          consultas automatizadas em \textit{websites}.

\end{itemize}

Considera-se de extrema importância a implementação de projetos desse
tipo pois o mesmo auxilia na compreensão e aplicação de Inteligencia 
Artificial em casos específicos.