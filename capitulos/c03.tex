\chapter{Proposta de experimento}

Para realizar o experimento será necessário treinar um modelo de rede
neural que seja capaz, ou esteja próximo, de decifrar um CAPTCHA. Para
isso serão efetuadas três etapas básicas e comuns quando se trabalha
com redes neurais. Primeiro será coletado o maior número possível de
imagens de CAPTCHA. Em seguida será gerado um dataset com as
características dessas imagens junto com a classe em que pertence. A
partir daí podemos realizar a configuração e treinamento da rede
neural. E por fim será testada a acurácia do modelo mediante imagens
de teste.

\section{Coleta de imagens}

A coleta de imagens será feita através de um script em Python que irá
acessar a página que possui uma imagem de CAPTCHA e assim fazer o
download da mesma. O script ficará em loop até que seja feita a coleta
de um número suficiente de imagens.

\section{Geração do Conjunto de dados}

Para criar os arquivos de conjunto de dados (ou ``dataset''), será
necessário utilizar o formato de arquivo {\bf HDF}, mais
especificamente a versão 5 (HDF5). O formato de arquivo de dados
hierárquico possibilita a manipulação de conjuntos de dados
extremamente grandes e complexos.

\subsection{Pré-processamento}

A fase de pré-processamento das imagens é mínima e é feita em tempo de
execução da geração do conjunto de dados.

\begin{itemize}
\item{\bf Escala de cinza}

Ao gerar um array representativo da imagem, apenas é considerado um
valor de escala de cinza da imagem, assim padronizando os valores de
intensidade de pixels entre 0 e 1.

\item{\bf Redimensionamento}

Ao gerar o array que representa a imagem, é feito um cálculo para
diminuir a imagem com base em uma escala. Essa escala será configurada
à partir de um valor padrão para a largura e altura das imagens.

\end{itemize}

\subsection{Conjunto de dados de teste}

Para o treinamento será necessário um conjunto separado para teste que
não possui nenhuma imagem presente no conjunto de treinamento. Para
isso será coletada uma amostra aleatória com cerca de 2\% do número de
imagens do conjunto de treinamento para geração do conjunto de dados
de teste.

\section{Treinamento}

Após gerado um arquivo contendo o conjunto de dados, é possível
trabalhar no treinamento do modelo da rede neural. Para isso será
usado o Framework {\bf TensorFlow}\cite{TensorFlow} destinado à
\textit{Deep Learning} e um script em Python que fará uso das funções
disponibilizadas pela biblioteca do TensorFlow. Assim realizando o
treinamento até atingir um valor aceitável de acerto no conjunto de
teste. O resultado do treinamento será um arquivo binário
representando o modelo que será utilizado para avaliação
posteriormente.

\subsection{Infraestrutura}

Com o intuito de acelerar o processo, foi utilizada uma
máquina com {\bf GPU} para o treinamento. A máquina foi adquirida em
uma \textit{Cloud} privada da AWS. A GPU utilizada se trata de uma
\textit{NVIDIA GRID K520} com 1.536 cores e 4GB de memória de vídeo. Como
processador a máquina possui um \textit{Intel Xeon E5-2670 (Sandy
  Bridge)} com 8 cores, e ainda possui 15GB de memória RAM
\cite{AWSGPU}\cite{GPUinstance}.

\section{Avaliação de acurácia}

Para a avaliação, uma nova amostra de imagens será coletada do mesmo
modo que foram coletadas as imagens para treinamento. Essa amostra
terá uma quantidade maior de imagens do que o conjunto de teste.

Com essas amostra de imagens, será feita a execução do teste do modelo
contra cada uma das imagens, assim armazenando uma informação de erro
ou acerto do modelo. Ao final da execução será contabilizado o número
de acertos e comparado com o número total da amostra de imagens para
avaliação. Resultando assim em uma porcentagem que representa a
acurácia do modelo gerado.

